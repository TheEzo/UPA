\documentclass[11pt,a4paper,titlepage]{article}
\usepackage[left=2cm,text={17cm,24cm},top=3cm]{geometry}
\usepackage[T1]{fontenc}
\usepackage[czech]{babel}
\usepackage[utf8]{inputenc}

\usepackage{graphicx}
\usepackage[ampersand]{easylist}
\usepackage{hyperref} % url
\usepackage{graphicx}
\bibliographystyle{czplain}

%uvozovky
\newcommand{\ceskeuvozovky}[1]{\quotedblbase#1\textquotedblleft}
\begin{document}

%\begin{titlepage}
%\begin{center}
%    {\LARGE\textsc{Vysoké učení technické v~Brně}}\\
%    \smallskip
%    {\Large\textsc{Fakulta informačních technologií}}\\
%    \bigskip
%    \vspace{\stretch{0.382}}
%    \LARGE{UPA - Ukládání a příprava dat}\\
%    \smallskip
%    \Huge{Projekt 1. část: návrh zpracování a uložení dat}\\
%    \vspace{\stretch{0.618}}
%\end{center}
%    {\Large Kapoun Petr, Bc. - xkapou04}\smallskip\\
%    {\Large Nováček Pavel, Bc. - xnovac16}\smallskip\\
%    {\Large Willaschek Tomáš, Bc. - xwilla00 \hfill \today }
%\end{titlepage}

%\tableofcontents
%\newpage

\part*{Projekt 1. část: návrh zpracování a uložení dat}
\begin{center}
Vysoké učení technické v~Brně\\
Fakulta informačních technologií\\
UPA - Ukládání a příprava dat\\
\today
\end{center}
\section{Zvolené téma}
04: COVID-19 (dr. Rychlý)
\section{Řešitelé}
\begin{itemize}
    \setlength\itemsep{0.3em}
    \item Bc. Kapoun Petr -- xkapou04,
    \item Bc. Nováček Pavel -- xnovac16,
    \item Bc. Willaschek Tomáš -- xwilla00.
\end{itemize}

\section{Zvolené dotazy a formulace vlastního dotazu}
\begin{itemize}
    \setlength\itemsep{0.3em}
    %\item dotaz A:
    \item \textbf{Dotaz A:} vytvořte popisné charakteristiky pro alespoň 4 údaje (např. věk, pohlaví, okres, zdroj nákazy) z datové sady COVID-19: Přehled osob s prokázanou nákazou dle hlášení krajských hygienických stanic (využijte krabicové grafy, histogramy, atd.).
    %\item v grafech zobrazte tempo změny počtů aktuálně nemocných (absolutní i procentuální přírůstek pozitivních případů a klouzavý průměr různých délek v různých časech)
    % \item dotaz B:
    %\item najděte skupiny nemocných s podobnými kritérii (např. podobný věk, místo, čas testů, atp.) a určete jejich vývoj v čase
    \item \textbf{Dotaz B:} určete vliv počtu nemocných a jeho změny v čase na sousední okresy (aneb zjistěte jak se šíří nákaza přes hranice okresů).
    %\item určete vliv epidemie COVID-19 na počet zemřelých v porovnání dle počtu nemocných, počtu hlášených úmrtí na nemoc COVID-19 a v porovnání s minulými lety
    %\item určete zpoždění změn přírůstku nemocných na přírůstek vyléčených v různých časech (aneb pokuste se odhadnout délku nemocnosti v různých časech)
    \item \textbf{Vlastní dotaz:} určete vliv věku na délku nemoci a úmrtnost.
    
\end{itemize}

\section{Stručná charakteristika zvolené datové sady}
% \textsc{Zde konkrétně popište jaké soubory budou představovat zdroj dat pro zvolené úlohy. Dále popište, jakým způsobem budou tato data získána a stručně charakterizujte strukturu souborů vybraných pro řešení projektu. Zaměřte se na části souborů, které jsou důležité pro zodpovězení zvolených dotazů.}

Základním zdrojem dat pro všechny dotazy jsou \textbf{otevřené datové sady COVID-19 v ČR}\cite{data_mzcr_covid_ofic} dostupné skrze veřejné API\footnote{\url{https://onemocneni-aktualne.mzcr.cz/api/v2/covid-19}} ve formátech JSON a CSV.

Datová sada \texttt{COVID-19: Přehled osob s prokázanou nákazou dle hlášení krajských hy\-gi\-e\-nic\-kých stanic (v2)} obsahuje následující charakteristiky o nakažených osobách:
\begin{itemize}
    \setlength\itemsep{0.3em}
    \item \texttt{datum} -- datum, kdy byla osoba pozitivně testována,
    \item \texttt{vek} -- věk nakažené osoby,
    \item \texttt{pohlavi} -- pohlaví nakažené osoby,
    \item \texttt{kraj\_nuts\_kod} -- identifikátor kraje podle klasifikace NUTS 3, ve kterém byla pozitivní nákaza hlášena krajskou hygienickou stanicí,
    \item \texttt{okres\_lau\_kod} -- identifikátor okresu podle klasifikace LAU 1,
    \item \texttt{nakaza\_v\_zahranici} -- příznak, zda došlo k nákaze mimo ČR,
    \item \texttt{nakaza\_zeme\_csu\_kod} -- identifikátor státu v zahraničí, kde došlo k nákaze (dvoumístný kód z číselníku zemí CZEM).
\end{itemize}

Další dvě použité datové sady jsou \texttt{COVID-19: Přehled vyléčených dle hlášení krajských hy\-gi\-e\-nických stanic} a \texttt{COVID-19: Přehled úmrtí dle hlášení krajských hygienických stanic}, které mají shodnou strukturu:
\begin{itemize}
    \setlength\itemsep{0.3em}
    \item \texttt{datum} -- datum vyléčení nebo úmrtí osoby,
    \item \texttt{vek},
    \item \texttt{pohlavi},
    \item \texttt{kraj\_nuts\_kod},
    \item \texttt{okres\_lau\_kod}.
\end{itemize}

Jelikož tyto datové sady používají identifikátory \textit{NUTS 3}\footnote{Nomenklatura územních statistických jednotek; úroveň 3 odpovídá krajům} pro kraje a \textit{LAU 1}\footnote{Local Administrative Units; úroveň 1 odpovídá okresům} pro okresy, využíváme dále číselníky od Českého statistického úřadu, které obsahují mapování těchto identifikátorů na odpovídající kraje a okresy a informace o nich.

\textbf{Číselník okresů}\footnote{Číselník okresů: \url{http://apl.czso.cz/iSMS/cisexp.jsp?kodcis=109&typdat=0&cisvaz=80007_210&datpohl=20.10.2020&cisjaz=203&format=2&separator=,}} a \textbf{číselník krajů}\footnote{Číselník krajů: \url{http://apl.czso.cz/iSMS/cisexp.jsp?kodcis=100&typdat=0&cisvaz=80007_885&datpohl=20.10.2020&cisjaz=203&format=2&separator=,}} jsou oba ve formátu CSV a obsahují:
\begin{itemize}
    \setlength\itemsep{0.3em}
    \item \texttt{KODJAZ} -- kód jazykové verze textů,
    \item \texttt{AKRCIS} -- akronym číselníku/klasifikace,
    \item \texttt{KODCIS} -- kód číselníku/klasifikace,
    \item \texttt{CHODNOTA} -- kód položky,
    \item \texttt{ZKRTEXT} -- zkrácený název položky,
    \item \texttt{TEXT} -- plný název položky,
    \item \texttt{ADMPLOD} -- počátek administrativní platnosti,
    \item \texttt{ADMNEPO} -- konec administrativní platnosti.
\end{itemize}

Zvoleným programovacím jazykem je \texttt{Python}\cite{python}. K získání dat použijeme standardní moduly jazyka \texttt{Python}\cite{python} 
\texttt{json}\footnote{Dokumentace modulu \texttt{json} jazyka \texttt{Python}\cite{python}: \url{https://docs.python.org/3/library/json.html}} a
\texttt{csv}\footnote{Dokumentace modulu \texttt{csv} jazyka \texttt{Python}\cite{python}: \url{https://docs.python.org/3/library/csv.html}}. K nahrání dat použijeme modul \texttt{elasticsearch}\footnote{Modul \texttt{elasticsearch} pro \texttt{Python}: \url{https://pypi.org/project/elasticsearch/}}.

\section{Zvolený způsob uložení uložení surových dat}
% \textsc{Zde stručně charakterizujte NoSQL databázi, která bude využita pro uložení zvolených zdrojových dat.}

Pro uložení dat jsme zvolili NoSQL řešení \texttt{Elasticsearch}\cite{elastic}\footnote{Dokumentace \texttt{Elasticsearch}\cite{elastic}: \url{https://www.elastic.co/guide/en/elasticsearch/reference/current/index.html}}. \texttt{Elasticsearch} je dokumentově orientovaný vyhledávací engine naprogramovaný v jazyce Java s použitím \texttt{Lucene}\cite{lucene}, díky kterému je velmi efektivní při komplexním vyhledávání na velkých objemech dat. Schéma struktur kolekcí odpovídá struktuře uložených dat ve vstupních datových sadách. Index je typu klíč-hodnota a pro každý vstupní soubor bude existovat právě jeden index. Klíč nabývá hodnoty \texttt{CHODNOTA} u číselníku okresů a krajů a ve zbývajících indexech se generuje automaticky, protože data neobsahují žádnou unikátní hodnotu.

% \url{https://blog.kuzzle.io/what-nosql-solution-should-you-choose-mongodb-elastisearch-oriendb}

%\newpage
\bibliography{literatura}

\end{document}

% \subsection{Kumulativní a denní přehledy dle KHS a laboratoří}
% 1. COVID-19: Celkový (kumulativní) počet provedených testů (v2) - \textsc{testy.(json/csv)}
% \begin{itemize}
%     \item datum
%     \item prirustkovy\_pocet\_testu
%     \item kumulativni\_pocet\_testu
% \end{itemize}

% 2. COVID-19: Celkový (kumulativní) počet osob s prokázanou nákazou dle krajských hygienických stanic včetně laboratoří (v2) - \textsc{nakaza.(json/csv)}
% \begin{itemize}
%     \item datum
%     \item prirustkovy\_pocet\_nakazenych
%     \item kumulativni\_pocet\_nakazenych
% \end{itemize}

% 3. COVID-19: Celkový (kumulativní) počet osob s prokázanou nákazou dle krajských hygienických stanic včetně laboratoří, počet vyléčených, počet úmrtí a provedených testů (v2) - \textsc{nakazeni-vyleceni-umrti-testy.(json/csv)}
% \begin{itemize}
%     \item datum
%     \item kumulativni\_pocet\_nakazenych
%     \item kumulativni\_pocet\_vylecenych
%     \item kumulativni\_pocet\_umrti
%     \item kumulativni\_pocet\_testu
% \end{itemize}

% 4. COVID-19: Základní přehled - \textsc{zakladni-prehled.(json/csv)}
% \begin{itemize}
%     \item datum
%     \item provedene\_testy\_celkem
%     \item potvrzene\_pripady\_celkem
%     \item aktivni\_pripady
%     \item vyleceni
%     \item umrti
%     \item aktualne\_hospitalizovani
%     \item provedene\_testy\_vcerejsi\_den
%     \item potvrzene\_pripady\_vcerejsi\_den
%     \item potvrzene\_pripady\_dnesni\_den
% \end{itemize}

% \subsection{Přehledy dle KHS}
% 5. COVID-19: Přehled osob s prokázanou nákazou dle hlášení krajských hygienických stanic (v2) - \textsc{osoby.(json/csv)}
% \begin{itemize}
%     \item datum
%     \item vek
%     \item pohlavi
%     \item kraj\_nuts\_kod
%     \item okres\_lau\_kod
%     \item nakaza\_v\_zahranici
%     \item nakaza\_zeme\_csu\_kod
% \end{itemize}

% 6. COVID-19: Přehled epidemiologické situace dle hlášení krajských hygienických stanic podle okresu - \textsc{kraj-okres-nakazeni-vyleceni-umrti.(json/csv)}
% \begin{itemize}
%     \item datum
%     \item kraj\_nuts\_kod
%     \item okres\_lau\_kod
%     \item kumulativni\_pocet\_nakazenych
%     \item kumulativni\_pocet\_vylecenych
%     \item kumulativni\_pocet\_umrti
% \end{itemize}

% 7. COVID-19: Přehled vyléčených dle hlášení krajských hygienických stanic - \textsc{vyleceni.(json/csv)}
% \begin{itemize}
%     \item datum
%     \item vek
%     \item pohlavi
%     \item kraj\_nuts\_kod
%     \item okres\_lau\_kod
% \end{itemize}

% 8. COVID-19: Přehled úmrtí dle hlášení krajských hygienických stanic - \textsc{umrti.(json/csv)}
% \begin{itemize}
%     \item datum
%     \item vek
%     \item pohlavi
%     \item kraj\_nuts\_kod
%     \item okres\_lau\_kod
% \end{itemize}
